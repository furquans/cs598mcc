% THIS IS SIGPROC-SP.TEX - VERSION 3.1
% WORKS WITH V3.2SP OF ACM_PROC_ARTICLE-SP.CLS
% APRIL 2009
%
% It is an example file showing how to use the 'acm_proc_article-sp.cls' V3.2SP
% LaTeX2e document class file for Conference Proceedings submissions.
% ----------------------------------------------------------------------------------------------------------------
% This .tex file (and associated .cls V3.2SP) *DOES NOT* produce:
%       1) The Permission Statement
%       2) The Conference (location) Info information
%       3) The Copyright Line with ACM data
%       4) Page numbering
% ---------------------------------------------------------------------------------------------------------------
% It is an example which *does* use the .bib file (from which the .bbl file
% is produced).
% REMEMBER HOWEVER: After having produced the .bbl file,
% and prior to final submission,
% you need to 'insert'  your .bbl file into your source .tex file so as to provide
% ONE 'self-contained' source file.
%
% Questions regarding SIGS should be sent to
% Adrienne Griscti ---> griscti@acm.org
%
% Questions/suggestions regarding the guidelines, .tex and .cls files, etc. to
% Gerald Murray ---> murray@hq.acm.org
%
% For tracking purposes - this is V3.1SP - APRIL 2009

\documentclass{acm_proc_article-sp}
\newcommand{\tup}[1]{\langle #1\rangle} 
\begin{document}

\title{Improving Live Migration: is it even worth it?}
% You need the command \numberofauthors to handle the 'placement
% and alignment' of the authors beneath the title.
%
% For aesthetic reasons, we recommend 'three authors at a time'
% i.e. three 'name/affiliation blocks' be placed beneath the title.
%
% NOTE: You are NOT restricted in how many 'rows' of
% "name/affiliations" may appear. We just ask that you restrict
% the number of 'columns' to three.
%
% Because of the available 'opening page real-estate'
% we ask you to refrain from putting more than six authors
% (two rows with three columns) beneath the article title.
% More than six makes the first-page appear very cluttered indeed.
%
% Use the \alignauthor commands to handle the names
% and affiliations for an 'aesthetic maximum' of six authors.
% Add names, affiliations, addresses for
% the seventh etc. author(s) as the argument for the
% \additionalauthors command.
% These 'additional authors' will be output/set for you
% without further effort on your part as the last section in
% the body of your article BEFORE References or any Appendices.

\numberofauthors{2} %  in this sample file, there are a *total*
% of EIGHT authors. SIX appear on the 'first-page' (for formatting
% reasons) and the remaining two appear in the \additionalauthors section.
%
\author{
% You can go ahead and credit any number of authors here,
% e.g. one 'row of three' or two rows (consisting of one row of three
% and a second row of one, two or three).
%
% The command \alignauthor (no curly braces needed) should
% precede each author name, affiliation/snail-mail address and
% e-mail address. Additionally, tag each line of
% affiliation/address with \affaddr, and tag the
% e-mail address with \email.
%
% 1st. author
\alignauthor
Author
\alignauthor
Author
}

\maketitle
\begin{abstract}
Virtualization technology is ubiquitous in most of today's datacenters and is an enabling technology for Infrastructure-as-a-Service Cloud Computing.  As the size of VM farms grows, concerns about the scalability of managing this infrastructure is growing.  Live migration techniques offer large flexibility gains at the cost of increased network traffic.  One solution to reduce this traffic involves exploiting the redundancy of data across virtual machines by migrating groups of similar virtual machines.  For this to be a worthwhile endeavor, we must first understand the nature of this redundancy.  The goal of this study is to investigate some possible use cases and determine the amount of memory redundancy that exists across VMs for these cases.  We develop a platform for evaluating this redundnacy and then utilize it to gain insight into the level of redundancy that can be exploited in these systems.
\end{abstract}

% A category with the (minimum) three required fields
%\category{H.4}{Information Systems Applications}{Miscellaneous}
%A category including the fourth, optional field follows...
%\category{D.2.8}{Software Engineering}{Metrics}[complexity measures, performance measures]

\section{Introduction}
Virtual Machine Migration \cite{live_gang}

\section{Redundancy Evaluation Architecture}

\subsection{Mathematical Definition for Redundancy}
In this section, we intend to give a succinct definition for what is mean by memory redundancy across Virtual Machines.  In all of our analysis, we define the similarity of N virtual machines' memory as a percentage value S:
\begin{equation}
S = \frac{|(P_A \mathbf{O} P_B ... \mathbf{O} P_N|}{\sum\limits_{i=1}^{N}|P_i|}
\end{equation}
where $P_i$ denotes the multiset of pages for VM $i$ (i.e. each element is the content of a specific page in memory) and we define the operator $\mathbf{O}$ as the ``pseudo-intersection'' of two multisets.  



If we let the multisets of pages be represented as sets of ordered pairs where $\tup{P,n}$ denotes that page $P$ appears $n$ times in the multiset,\footnote{The authors would like to gratefully acknowledge the help of Asaf Karagila in formalizing this definition.}
then we can say:
\begin{equation}
A\mathrel{\mathbf{O}}B\equiv\{\tup{x,i+j}\mid\tup{x,i}\in A\land\tup{x,j}\in B\}.
\end{equation}\label{eqn:s}

The operator $\mathbf{O}$ is easily illustrated by a simple example rather than formal definition:

\begin{align*}
\textrm{Let } & A = \{1,1,2,5\},~B = \{1,2,3\}\\
\textrm{Then } &{\cal A} = \{\tup{1,2}, \tup{2,1}, \tup{5,1}\}\\
& {\cal B} = \{\tup{1,1}, \tup{2,1}, \tup{3,1}\}\\
{\cal A} \mathbf{O} {\cal B} &= \{\tup{1,3}, \tup{2,2}\} = \{1,1,1,2,2\}
\end{align*}

The example above may not be 100\% rigorous as we liberally switch between multiset and ordered pair representations, but the explanation still expresses the quantity we defined.  This metric is extremely useful when determining the amount of redundancy in a virtual machine as it quantifies the number of pages that are present across all VMs.

It should be noted than since many zero pages exist and compression techniques could be used to avoid sending these pages altogether\cite{live_adaptive_compress}, we define another metric $S_z$:
\begin{equation}
S_z = \frac{S-N_z}{\sum\limits_{i=1}^{N}|P_i|-N_z}
\end{equation}
where $N_z$ is the total number of zero pages in all VMs:
\begin{equation}
N_z = \sum\limits_{i=1,Z\subseteq P_i}|P_i\mathbf{O}Z|-1.
\end{equation}\label{eqn:sz}
In the equation above, $Z$ is the zero page and the $-1$ balances out the extra count we get from the $Z$ on the right-hand-side of $\mathbf{O}$.

\section{Optimistic Upper Bounds}
Before proceeding with applied use cases, we decided to get an upper bound for redundancy by testing a subset of ``ideal'' situations.  When asking the question
``is it even worth it?'' we wanted to run a test against very simple use cases to ensure that we should even continue with the project.  To do this, we used an Openstack cloud to generate multiple virtual machines and test their


%Here, we compare two basic cases. On the left, we have one cloud image that is instantiated multiple times. On the right, we test VMs of different operating systems against each other. The half above the diagonal does not include zero pages and the lower half includes zero pages. As can be expected, not much redundancy exists across different OS versions (though this may change with real applications). All VMs have 1GB of memory.


\section{Stability Over Time}
Folding at home, surprising that it was stable until...

\section{Client-Server}

\section{Hadoop}

\section{Stability Over Time}

\section{Conclusions}

\bibliographystyle{abbrv}
\bibliography{cs598mcc_paper}

\end{document}
